% !TeX encoding = UTF-8
% !TeX root = MAIN.tex

{%
	\selectlanguage{naustrian}
	\chapter*{Kurzfassung}
	
		Diese Arbeit hat zum Ziel, einen Überblick über die Methode der Modified Nodal Analysis (MNA) und deren Diskretisierung zu geben, wobei der Fokus auf einfacheren Netzwerken liegt, die nur aus Widerständen, Kondensatoren, Induktivitäten, Spannungsquellen und Stromquellen bestehen, sogenannte RLC-Netzwerken. Wir beginnen mit der Beschreibung elektrischer Schaltungen und ihrer Komponenten sowie physikalischer Gesetze. Dazu formulieren wir die Topologie der elektrischen Schaltung mathematisch, anschließend leiten wir die Formulierung der MNA her. Es handelt sich bei der Struktur des resultierenden Systems um eine Differential-Algebraische Gleichung. Diese Struktur analysieren wir weiter und wenden die Ergebnisse auf die MNA an. Abschließend betrachten wir Möglichkeiten, dieses System mittels \emph{Mehrschrittmethoden} zu diskretisieren und diskutieren deren Konvergenzraten.
}

{%
	\selectlanguage{english}
	\chapter*{Abstract}
			
		In this talk we give an overview of the Modified Nodal Analysis Method (MNA) and its discretization by focusing on simpler networks only consisting of resistors, capacitors, inductors, voltage sources and current sources, so-called RLC-networks. We start by describing electrical circuits and their components as well as the relevant physical laws. To this end, we first describe the topology of the electrical circuit mathematically and then derive the MNA formulation. The structure of the resulting system is a Differential Algebraic Equation, which we continue to analyze further. These results are then applied to the MNA formulation. Finally, we discuss the discretization of such systems using \emph{Multistep Methods}, examine their convergence rates, and provide illustrative examples to support the theoretical results.
		
		
		
		
		%%%%%%%%%%%%%%%%%%%%%%%%%%%%%%%%%%%%%%%%%%%%%%%%%%%%%%%%%%%%%
		-numerische resultate um convergence rates zu illustraten
}