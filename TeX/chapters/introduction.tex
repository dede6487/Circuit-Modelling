\chapter{Introduction}

	Circuit modelling is
	
	over the last x years
	
	connecting to this ... MNA 
	
	
	
	The thesis is organized as follows. In Section \ref{sec:formulating a mathematical model} ... general derivation. first topology ... Resulting system of equations is DAE, then analyzed in Section \ref{sec:Differential Algebraic equations}. Section \ref{sec:index analysis of the modified nodal analysis} then applies results of the previous section to the specific problem. finally Section \ref{sec:Numerical Solutions} describes a way to discretize and applies to specific examples.




	This chapter should include information about what circuit modelling wants to achieve as well as giving an overview of what this bachelor-thesis is about.

What is this thesis about? \newline
	Modelling and numerically solving systems that arrise from electrical circuits with RLC elements. Furthermore it will briefly discuss on expanding this baseline with more complicated electrical components.
What is the goal of this thesis?\newline
	The goal of this thesis is to give insight into the state of the art of circuit modelling. It aims to elaborate on the underlying concepts of MNA as well as on the most commonly used numerical methods.
	
	Kapitel zwei gibt Überblick über die Physikalischen Zusammenhänge und deren Mathematische Beschreibung. Anschließend wird ein Mathematisches Modell nach der Methode der ''Modified Nodal Analysis`` erstellt. Im dritten Kapitel analysieren wir die Systeme, welche durch diese Analyse entstehen, wobei das vierte Kapitel eine wichtige Eigenschaft dieser Systeme genauer untersucht. Das letzte Kapitel beschäftigt sich mit der Numerischen Lösung dieser Systeme.  
	
	je etwas mehr Text zu den Kapiteln (Struktur der Kapitel)