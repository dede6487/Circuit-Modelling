\chapter{Introduction}

	Circuit modelling plays an essential part the design process of modern devices. Without the ability to model electrical networks...
	
	Beginning in the last century and still ongoing research has made many advancements in understanding electrical networks and in developing more sophisticated methods of analyzing and discretizing electrical networks. In the late 20th century, this lead to the introduction of the so called \emph{Modified Nodal Analysis} which overcame the previously wide spread \emph{Nodal Methods} limitations and was in turn widely used. The resulting system of equations of this Modified Nodal Analysis are of the form of so called \emph{Differential Algebraic Equations}. These equations themselves bring other challenges when it comes to discretization. 
	
	This thesis aims to give an overview of the Modified Nodal Analysis Method and its discretization by focusing on simpler networks only consisting of resistors, capacitors, inductors, voltage sources and current sources (so called RLC-networks). It is organized as follows. In Section \ref{sec:formulating a mathematical model} we start by describing electrical circuits and their components as well as physical laws. We first describe the topology of the electrical circuit mathematically, then we derive the Modified Nodal Analysis formulation. As already mentioned, the structure of the resulting system is a Differential Algebraic Equation. Section \ref{sec:Differential Algebraic equations} analyses this structure further. In Section \ref{sec:index analysis of the modified nodal analysis} we then apply the results of the previous section to the Modified Nodal Analysis. Finally in Section \ref{sec:Numerical Solutions} we find ways to discretize this system by means of \emph{Multistep Methods} and discuss their convergence rates.
