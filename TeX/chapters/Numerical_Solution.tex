\chapter{Numerical Solutions}
This chapter focuses on the numerical solution of the mentioned systems. (based on the DAE-lecture)
It will include Code that illustrates some of the shown procedures.

We will first focus on the general methods used for solving a more general problem
\begin{align}
	\label{general numerical problem}
	y'(t) &= f(t,y), \quad t \in [t_0, t_l], \\
	y(t_0) &= y_0.
\end{align}


We will presume that the function $f(t,y)$ is continuous and Lipschitz, thus for every $y_0$ it is uniquely solvable in $[t_0, t_l]$.

\begin{figure}[H]
	\centering
	\includegraphics[width=0.7\linewidth]{screenshot015}
	\caption{}
	\label{fig:screenshot015}
\end{figure}

from book circuit modelling 35-47

\section{Single-Step-Methods}
	page 46 modelling circuit
	
	page 20 num gew dgl
	
	\begin{definition}
		A numerical method to approximate a differential equation \ref{general numerical problem} on a time-grid $t_0,...,t_l$ with the intermediate values $y_0,...,y_l$ is called a single-step method if it is from the form
		\begin{equation}
			\label{single-step method}
			y_{j+1} = y_j + h_j \phi(t_j,y_j, y_{j+1},h_j).
		\end{equation}
		With the \emph{procedural function} $\phi$. If $\phi$ is not dependent on $y_{j+1}$ then the method is called \emph{explicit}, otherwise it is called \emph{implicit}.
	\end{definition}

	\subsection{Consistency and Convergence}
	
	\begin{definition}
		Let $\tilde{y}_{m+1}$ be the result of one step of \ref{single-step method} with the exact start-vector $y_m = y(t_m)$ then
		\begin{equation}
			\label{local discretization error single step}
			le_{m+1} = le(t_m+h) = y(t_{m+1}) - \tilde{y}_{m+1}, \quad m = 0,...,N-1
		\end{equation}
		is called the \emph{local discretization error} of the single step method at the point $t_{m+1}$.
	\end{definition}


\section{Multistep-Methods}
	based on chapter 4 of book num gew dgl steif nichtsteif \newline
	Linear multistep methods use approxiamtions $u_{m+l}$ along the gridpoints $t_{m+l}, \quad l=0,1,...,k-1$ to calculate the new approximation $u_{m+k}$ at $t_{m+k}$. WE will first discuss topics related to the order of the methods depending on its parameters, stability and convergence.
	
	\begin{definition}
		For given $\alpha_0, ..., \alpha_k$ and $\beta_0, ..., \beta_k$ the iteration rule
		\begin{equation}
			\label{linear-multistep-method}
			\sum_{l=0}^{k} \alpha_l u_{m+l} = h \sum_{l=0}^{k} \beta_l f(t_{m+l}, u_{m+l}), \quad m=0,1,...,N-k
		\end{equation}
		is called a \emph{linear multistep method} (linear k-step method). It is always assumed that $\alpha_k \neq 0$ and $|\alpha_0| + |\beta_k| > 0$. If $\beta_k=0$ holds, then the method is called explicit, otherwise implicit.
	\end{definition}
	
	\begin{figure}[H]
		\centering
		\includegraphics[width=0.7\linewidth]{screenshot010}
		\caption{}
		\label{fig:screenshot010}
	\end{figure}
	
	A linear multi-step method consists of two parts:
	\begin{enumerate}
		\item In the \emph{starting-phase} approximations $u_1,...,u_{k-1}$ for the first $k-1$ gridpoints $t_l = t_0+th, l=1,...,k-1$ are calculated using a single-step method. For example using an explicit Runge-Kutta Methodor a multi-step method with fewer steps.
		
		\item  In the \emph{run-phase} the multi-step formula is used to determine new approximations $u_{m+k}$ for the gridpoint $t_{m+k}$
	\end{enumerate}
	
	For theoretical analysis of the multi-step methods we consider the generating polynomials
	\begin{equation}
		\rho(x) := \sum_{l=0}^{k} \alpha_l x^l
	\end{equation}
	\begin{equation}
		\sigma(x) := \sum_{l=0}^{k} \beta_l x^l
	\end{equation}
	
	\begin{figure}[H]
		\centering
		\includegraphics[width=0.7\linewidth]{screenshot013}
		\caption{}
		\label{fig:screenshot013}
	\end{figure}
	
	
	\subsection{Consistency and order}
	local discretization error - def 4.2.2
	\begin{definition}
		Let $\tilde{u}_{m+k}$ be the result of one step of the multi-step method \ref{linear-multistep-method} with the start-vectors $u_m, u_{m+1}, ..., u_{m+k-1}$ lying on the exact solution $y(t)$ of the problem \textbf{reference}. This means
		\begin{displaymath}
			\alpha_k \tilde{u}_{m+k} = \sum_{l=0}^{k-1} \left( h \beta_l f(t_{m+l}, y(t_{m+l})) - \alpha_l y(t_{m+l}) \right) + h \beta_k f(t_{m+k}, \tilde{u}_{m+k}) .
		\end{displaymath}
		Then
		\begin{displaymath}
			le_{m+k} = le(t_{m+k}) = y(t_{m+k}) - \tilde{u}_{m+k}, \quad m=0,1,...,N-k
		\end{displaymath}
		is called the \emph{local discretization error} (local error) of the linear multi-step method \ref{linear-multistep-method} at the point $t_{m+k}$.
	\end{definition}
	
	We will assign the linear difference opreator
	\begin{equation}
		L[y(t),h] = \sum_{l=0}^{k} \left( \alpha_l y(t+lh) - h \beta_l y'(t+lh) \right)
	\end{equation}
	to the local discretization error. Using this we gain the following definition.
	
	\begin{definition}
		A linear multi-step method is called \emph{preconsistent} if for all functions $y(t) \in C^1[t_0,t_l]$
		\begin{displaymath}
			\lim\limits_{h \to 0} L[y(t),h]=0
		\end{displaymath}
		holds. It is called \emph{consistent}, if for all functions $y(t) \in C^2[t_0,t_l]$
		\begin{displaymath}
			\lim\limits_{h \to 0} \frac{1}{h} L[y(t),h] = 0
		\end{displaymath}
		holds. It has the \emph{consistency order p}, if for all functions $y(t) \in C^{p+1}[t_0, t_l]$
		\begin{displaymath}
			L[y(t),h] = \mathcal{O}(h^{p+1}) \quad \text{for} \quad h \to 0
		\end{displaymath}
		holds.
	\end{definition}
	
	\begin{figure}[H]
		\centering
		\includegraphics[width=0.7\linewidth]{screenshot014}
		\caption{}
		\label{fig:screenshot014}
	\end{figure}
	
	
	
	\subsection{Convergence and stability}
	
	\begin{definition}
		A linear multi-step method is called \emph{zero-stable} if all solutions of the difference equation
		\begin{displaymath}
			\sum_{l=0}^{k} \alpha_l u_{m+l} = 0
		\end{displaymath}
		are bounded.
	\end{definition}
	
	\begin{theorem}
		A linear multi-step method is zero-stable, if and only if the polynomial $\rho(x)$ fullfills the root-condition, this means:
		\begin{enumerate}
			\item All roots $\bar{x}$ of $\rho(x)$ are within the unit-circle $|\bar{x}| \leq$ in the complex plane.
			\item All roots $\bar{x}$ with $|x| = 1$ are singular.
		\end{enumerate}
	\end{theorem}
	
	
	\textbf{from circuit book below, above from modelling book}

\section{Implicit linear multi-step formulas}
These kinds of multi-step methods are conventionally used to numerically solve the systems obtained using modified nodal analysis. 

The conventional approach can be split into three main steps:
\begin{enumerate}
	\item Computation of consistent initial values
	\item numerical integration based on multi-step schemes
	\item transformation of the DAE into a nonlinear system and its numerical solutioon by Newton's procedure (?????????????????will not be discussed further because not very specific)
\end{enumerate}

\textbf{Consistant initial values} 
\begin{figure}[H]
	\centering
	\includegraphics[width=0.7\linewidth]{screenshot009}
	\caption{}
	\label{fig:screenshot009}
\end{figure}

\textbf{Nunmerical integration}.


\subsection{BDF schemes and trapezoidal rule}